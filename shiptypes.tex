t

\section{Common ship body and propulsion types}

Throughout history, countless ship types have been brought to life, due to the various and ever-changing environmental challenges they had to overcome. Therefore they can be grouped or ordered several different ways, based on their shape, propulsion, area of application, origin and so forth.

\subsection{Hull types}

The lore of shipbuilding/footnote{Just like many other scholars of ancient history, early masters of shipbuilding were considered artists \cite{Art_of_shipbuilding}} is centered around the design of the hull, and it's probably a sensible way to start categorizing ships.
The fundamental concept of a ship hull design is the low resistance along the direction of forward motion. A lower drag results in faster top speed and lower energy requirements. Lateral shape is only mildly important in the design of self propelled ships, however sailing with side-wind is only possible by balancing the front and lateral resistances carefully\cite{vitorlazas}.

\paragraph{Lateral surface} affects the ability of the ship to sail and turn. A high lateral surface results in good wind-response but generally decreases the turning agility and increases the frontal surface and friction.

\subsection{Stability of a ship} 

Stability is the ability of the vessel to return to it's previous position\cite{stability}. This righting movement is the result of the relationship between the \textbf{Center of Mass} (CoG), and the \textbf{Center of Buoyancy} (CoB) of the ship. The stability can be divided to three different movements, independent from each other.

\paragraph{Static stability} is the stability at rest, without any external forces. In contrary to the CG, which is a point fixed to the body of the ship\footnote{Unless the ship features moving ballast, like the sailor of a dinghy}, the CoG constantly changes it's position as the ship heels or trims. If the CoB and CoG don't align vertically, the ship keeps changing it's attitude, until they do.

\paragraph{Form stability} As the boat moves, some area submerges, some reveals from the water. As the hull is heeling, the CoB moves, and depending on the form stability the boat will either overturn, or enough counter-tourqe will build up to reverse the tilting motion.

\paragraph{Dynamic stability} is generated when the ship is moving, and the effect is increased with speed. The dynamic stability can either increase or decrease the overall stability, but most modern boats are considerably more stable at higher speeds, due to special hull shaping. As the pressure increases under the hull, the dynamic forces will become stronger compared to the effects of buoyancy, and a round-bottomed hull can become unstable, as the width and shape of the hull beneath the water is significantly narrower than during static buoyancy.

It's notable, that even though a boat with high form and dynamic stability can offer a pleasant still water comfort, also provides a very rough ride at higher speeds, due to the quick hull reactions. On the contrary, a smooth wave-cutter promises constant swaying and vertical movement in a harbor.

One of the earliest efforts to increase stability was the extra \textbf{Ballast} added to the bottom of the ship, ranging from a couple of rocks to a keel mounted metal fin. The principle of the ballast is to lower the center of gravity. Many modern sailboats feature a lower CoG than the CoB therefore becoming immune to capsizing. Even after a complete turnover, the boat will eventually return to it's original state\footnote{Given that no structural damage happened during this undesirable event}.
Another, but considerably different effort to ensure stability is a wide \textbf{Beam}ed hull. While the ballast provides good ultimate stability, it lacks the initial stability, and the body can gather up quite some tilt until the gravity starts working. Contrary to the ballast, a wide beam provides good initial stability but does not provide the righting movement after a certain point. However,  wide beam monohulls can be dangerous, because it reaches the peak of initial stability quickly and the ship is more likely to capsize, but multihulls\footnote{Catamarans, trimarans} can boast exceptional stability qualities.
The response speed of the ship can be adjusted by the placement of equipment and cargo on the ship. By moving weight away from the centerline horizontally, the increased \textbf{inertia} dampens the quick movements caused by high initial stability.


 
\paragraph{The length of the ship} determines the \textbf{longitudinal stability}. (...)

\paragraph{The lateral stability} is the response of the ship to the tilt of the body, resulting in what is called the \textbf{righting movement}. This A wider and flatter shape increases the quick response (dynamic stability), while a weighted keel or ballast generates slower, but constant response (static stability). Special parts with high lateral surface, like the keel can increase the rolling resistance, however they do not generate stability by themselves.

(..figures tilt degree<->shape and weight stability forces plotted..)


A weighted keel can significantly move the center of gravity of the ship. A well designed modern keeled sailing boat is therefore immune to capsizing, which is a very satisfying feature in a remote storm. This is called a self-righting movement.

The dynamic stability of the ship can be dramatically increased without sacrificing the frontal resistance using multi-hull designs. Catamarans and trimarans are getting more and more popular, though they feature ancient design by Polynesian natives. Multihull ships rely their righting movement on the spacing between the hulls.
Advantages of the multihull design are (...), however (...)


These are the central problems of the design of sailing ships, and many centuries had to pass until humanity discovered the way to sail sideways to the wind, and even towards the wind in some degree.

\subsection{Propulsion types}

Modern ships have introduced remarkably advanced propulsion technologies, and even more are still in development. This section will give a quick overview about the most widespread possibilities, but the thesis will analyze only some of them.

\paragraph{Wind-powered}

Contrary to the belief, sailing ships are not the only wind powered ships currently existing, although they proved to be the most efficient. However, Rotor ships are still manufactured, mostly as part of a hybrid propulsion system.

Wrapfig Here.

Sailing ships will be described in detail later.

\paragraph{Mechanized propeller propulsion}

For the sake of clarity, we will not distinguish the different types of propulsion by power source, only based on the actuation. Steam and Gas turbines, Diesel engines and Nuclear engines both work on similar principle of powering the ship drive, which is usually one or more propellers. During the Steam era notable means of propulsion were the Paddle wheels, water caterpillars and also Horse-powered ships, but these were less common.

\subsection{Considerations}