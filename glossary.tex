\makeglossaries

Through the thesis I've tried to express the ideas and concepts as clearly as possible, but there are a good deal of, mostly nautical expressions, that consistently omitting or replacing them would do to the intelligibility more harm than good. I have often found myself wondering about the meaning of certain terms, and have decided to include a glossary. By any means, skip this part for now, but remember to check back if you have found yourself facing a sentence filled with marine, military or engineering notions.

\newglossaryentry{aft}
{
  name=Aft,
  description={Backwards direction on the ship along the longitudinal axis (e.g. Aftmost cannon)}
}
\newglossaryentry{beam}
{
  name=Beam,
  description={Width of the ship at the widest part. Alternative meaning is a direction perpendicular to the body of the ship.}
}
\newglossaryentry{bow}
{
  name=Bow,
  description={The foremost part of the boat that touches the water.}
}
\newglossaryentry{Bulbous_bow}
{
  name=Bulbous bow,
  description={}
}
\newglossaryentry{fore}
{
  name=Fore,
  description={Forward direction on the ship along the longitudinal axis (e.g. Foremost cannon)}
}
\newglossaryentry{heave}
{
  name=heave,
  description={Translational motion in the vertical direction}
}
\newglossaryentry{hydrodynamics}
{
  name=Hydrodynamics,
  description={The study of fluid motion}
}
\newglossaryentry{pitch}
{
  name=Pitch,
  description={Rotation around the lateral axis}
}
\newglossaryentry{port}
{
  name=Port,
  description={Left side of the ship, indicated with red light}
}
\newglossaryentry{aft}
{
  name=Roll,
  description={Rotation around the longitudinal axis}
}

Roll: Rotation around the longitudinal axis
Surge: Translational motion in the longitudinal direction
Starboard: Right side of the ship, indicated with green light
Stern \ T�k�r: The aftmost part of the boat that touches water
Sway: Translational motion in the lateral direction
Yaw: Rotation around the vertical axis


